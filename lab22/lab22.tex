\documentclass[fleqn, 12pt]{article}
\usepackage[a4paper, margin={10mm, 10mm}, includefoot]{geometry}

\usepackage[english, russian]{babel}
\usepackage{indentfirst}
\usepackage{mathtools}
\usepackage{setspace}

\author{Максим Гига}
\title{Конспект по математическому анализу, лекция \textnumero 12}

\begin{document}
\onehalfspacing
\maketitle

\section{Определение дифференциала}

\textbf{Определение:}
Пусть функция $f: U(x_0) \rightarrow R$ определена в $U(x_0)$,
\textit{приращение} $\Delta f(x_0, \Delta x)$ функции $f$
при $\Delta x \to 0$ можно представить в виде
\begin{math}
	\Delta f(x_0, \Delta x) = f(x_0 + \Delta x) - f(x_0) = A * \Delta x + o(\Delta x),
\end{math}
где
\begin{math}
	A = const, \quad \lim_{x \to 2} f(x) = 5 \Rightarrow
\end{math}

\(f\) называется \textit{дифференцируемой} в точке
\(x_0\), \(df = fy = A \Delta x = A d x\) -
\textit{дифференциал} функции, \(\Delta x = d x\) - приращение аргумента.

\textbf{Определение:}
Дифференциал - главная линейная (относительно приращения аргумента) часть
приращения функции.

\section{Критерий дифференцируемости}

\textbf{определение:}
\(f(x)\) дифференцируема в точке \(x_0 \Leftrightarrow \exists f'(x_0) \)

\textbf{доказательство:}
\begin{displaymath}
	1. \Rightarrow
	\Delta f = A \Delta x + o(\Delta x), \quad \Delta x \to 0 \Rightarrow
	\lim_{\Delta x \to 0} \frac{\Delta f}{\Delta x} = A = f'(x_0)
\end{displaymath}
\begin{displaymath}
	2. \Leftarrow
	\exists f'(x_0) \Rightarrow \frac{\Delta f}{\Delta x} =
	f'(x_0) + o(\Delta x) \Rightarrow \Delta f = f'(x_0) \Delta x + o(\Delta x),
	d f = f' dx
\end{displaymath}

\section{Геометрический смысл производной и дифференциала}

Пусть функция \(y = f(x)\) - дифференцируемая на
\((a, b),
\quad x, x_0 \in (a, b),
\quad y_o = f(x_0), \quad M_0 = (x_0, y_0),
\quad x_0 + \Delta x = x,
\quad f(x_0 + \Delta x) = y
\quad M = (x, y)\)

Уравнение секущей, проходящей через точки \(M_0, M: \\
y - y_0 = k(\Delta x)(x - x_0)\), где \(k(\Delta x)\) - угловой коэффициент \\
\(\frac{\Delta y}{\Delta x} = k(\Delta x)
\quad \Delta x \to 0
\quad \Delta y \to 0\)

В силу непрерывности функции \(f\) малому приращению аргумента
соответствует малое приращение функции:
\(\Delta x \to 0 \Rightarrow \Delta y \to 0\)
Расстояние между точками \(M, M_0
\quad \rho (M, M_0) = \sqrt{\Delta x^2 + \Delta y^2} \to 0 \Rightarrow M \to M_0 \)
при \(\Delta x \to 0\)

\textbf{Определение:} Предельное положение секущей (если этот предел существует)
называется \textit{касательной} к кривой в данной точке
$k_0 = \lim_{\Delta x \to 0} k(\Delta x) =
	\lim_{\Delta x \to 0} \frac {\Delta y}{\Delta x} = y'(x_0)$ -
угловой коэффициент касательной

\textbf{Уравнение касательной:} $y - y_0 = y'(x_0)(x - x_0)$

\textbf{Геометрический смысл производной:}
Тангенс угла наклона касательной к графику функции в данной точке.
$k_0 = y'(x_0) = tan \alpha$, где $\alpha$ - угол наклона к оси $Ox$ абцисс.

\textbf{Геометрический смысл дифференциала:}
Приращение ординаты касательной к графику функции
в соответствующей точке.

\textbf{Механический смысл производной:}
Скорость изменения некоторой величины есть ее производная.

\section{Правила дифференцирования}
\textbf{Утверждение:} Если существуют производные $f', g' \Rightarrow$

\begin{enumerate}
	\item $(C)' = 0 $
	\item $(C f)' = C f' $
	\item $(f \pm g)' = f' \pm g' $
	\item $(f g)' = f'g + g'f $
	\item $(\frac{1}{g(x)})' = -\frac{1}{g^2(x)}g'(x), g \ne 0 $
	\item $(\frac{f}{g})' = -\frac{f'g - g'f}{g^2}, g \ne 0 $
\end{enumerate}

\section{Производные и дифференциалы высших порядков}
\textbf{Определение:} \(y = f(x), \quad x \in (a, b)
\quad \exists y' = f'(x) = z(x), \quad x \in (a, b). \)

Если $\exists z'(x) = (y')' = \frac{d}{dx}(\frac{dy}{dx}) = \frac{d^2 y}{d x^2}$,
то она называется \textit{второй производной функции} $y$, а $y$ -
\textit{дважды дифференцируемой}

\(y^{(n)}(x) \coloneqq \frac{d}{dx}(y^{(n-1)}(x))\)

\textbf{Утверждение:} если $\exists f^(n), g^(n) \Rightarrow $
\begin{itemize}
	\item $(f \pm g)^{(n)} = f^{(n)} \pm g^{(n)}$
	\item $(f g)^(n) x = \sum_{i=0}^{n} C^i_n f^{(i)} g^{(n - 1)} $ - формула Лейбница, где
	      $C^i_n = \frac{n!}{(n - i)!\,i!}$
\end{itemize}

\section{Производная параметрически заданной функции}
\textbf{Определение:}

\[
	\begin{cases}
		x = x(t) \\
		y = y(t)
	\end{cases}
	t \in (\alpha; \beta)
\]

Пусть $x \uparrow\uparrow (\downarrow\downarrow) $ строго монотонна и непрерывна в
$U(t_0) \in (\alpha, \beta) \Rightarrow \exists t = t(x) $ обратная к $x(t)$ и
$\exists y(t(x))$. Эта функция называется параметрически заданной системой
уравнений, $t$ параметр.

\textbf{Утверждение:}
\begin{enumerate}
	\item \begin{displaymath}
		      \exists x'|_{t=t_0} \ne 0,
		      \exists y'|_{t=t_0} \Rightarrow
		      \exists y_x'(x_0) = \frac{y_t'}{x_t'}\Bigr|_{t=t_0}
	      \end{displaymath}
	\item \begin{displaymath}
		      \exists x''|_{t=t_0}
		      \exists y''|_{t=t_0} \Rightarrow
		      \exists y_{xx}'' = \frac{y_{tt}'' x_t' - y_t' x_{tt}''}{(x_t')^3}\Bigr|_{t=t_0}
	      \end{displaymath}
\end{enumerate}

\section{Производные высших порядков от обратных функций}
\textbf{Утверждение:} Пусть $y : \mathring{U}(x_0) \rightarrow R, \quad y = y(x),
	\quad y \uparrow\uparrow (\downarrow\downarrow) $ - строго монотонна в $U(x_0)$ и
непрерывна, $\exists y', y''$ в
$x_0, y'(x_0) \ne 0 \Rightarrow
	\exists x = x(y), \exists x_y' = \frac{1}{y_x'} \quad
	\exists x_{yy}'' = - \frac{y_{xx}''}{(y_x')^3}$ в $y_0 = y(x_0)$

\textbf{Доказательство:}
$x_{yy}'' = (x_y')_y' = (\frac{1}{y_x'})_y' =
	(\frac{1}{y_x'})_x' \frac{dx}{dy} =
	- \frac{y_{xx}''}{(y_x')^2 y_x'} =
	- \frac{y_{xx}''}{(y_x')^3} $

\end{document}
