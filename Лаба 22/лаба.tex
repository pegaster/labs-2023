\documentclass[a4paper,12pt]{article}
\usepackage{graphicx} % Required for inserting images

% подключаем русский
\usepackage[T2A]{fontenc}
\usepackage[utf8]{inputenc}
\usepackage[english,russian]{babel}
\usepackage[margin=1in]{geometry}

% подключаем математику
\usepackage{amsmath,amsfonts,amssymb,amsthm,mathtools}
\author{Мустафаев Алим. Группа М8О-113Б-23}
\title{Лабораторная работа по теме \LaTeX{}}


\begin{document}
\maketitle
\newpage
\begin{center}
Глава 3. \textit{Вычеты функций}
\end{center}
\noindent\rule{\textwidth}{1pt}
\section{Теорема Коши о вычетах. Приложение вычетов к вычислению определенных интегралов. Суммирование некоторых рядов с помощью вычетов}
\subsection{Теорема Коши о вычетах}
\textbf{Теорема.} \textit{Если функция f(z) является аналитической на границе C области D и всюду внутри области, за исключением конечного числа особых точек $z_1, z_2,..., z_n$, то}
\[ \int\limits_C f(z)dz = 2\pi i\sum_{k=1}^n res f(z_k).\]
\textbf{Пример 1.} Вычислить интеграл

\[\int\limits_{|z|=4} \frac{e^z - 1}{z^2 + z} dz.]\]

\textit{\underline{Решение.}} В области |z| < 4 функция $f(z) = \frac{e^z-1}{z^2+z}$ аналитична всюду, кроме z = 0 и z = -1.

По теореме Коши о вычетах
\[\int\limits_|z|=4 \frac{e^z-1}{z^2+z} dz \approx 2\pi (res f(0) + res f(-1)) \]
Точка z = 0 есть устранимая особая точка функции f(z), ибо
\[\lim_{x \to 0} \frac{e^x - 1}{z(z+1)} = 1.\]
Поэтому $res f(0) = 0$. Точка z = -1 -- полюс первого порядка,
\[ resf(-1) = \lim_{x \to -1} 
\left( \frac{e^z - 1}{z(z + 1)} (z+1) \right) = 1 - e^{-1}\]
Таким образом,
\[ \int\limits_{|z|=4} \frac{e^z-1}{z^2+z} dz = 2\pi i(1-e^{-1}).\]

\textbf{Пример 2.} Вычислить интеграл
\[ \int\limits_{|z|=2} \tg zdz.\]

\textit{\underline{Решение}}. В области D: |z| < 2 функция $f(z) = \tg z $ аналитична всюду, кроме точек $z = \frac{\pi}{2}$ и $z = -\frac{\pi}{2}$, являющихся простыми полюсами. Все другие
\newpage
\begin{center}
\S11. \textit{Теорема Коши о вычетах}
\end{center}
\noindent\rule{\textwidth}{1pt}
особые точки $z_k = \frac{\pi}{2} + k\pi $ функции $f(z) = \tg z$ лежат вне области D и поэтому не учитваются.

Имеем
\[ res f\left(\frac{\pi}{2} \right) = \left.\frac{\sin z}{(\cos z)} \right|_{z = \pi/2} = -1, res f\left(-\frac{\pi}{2} \right) = \left.\frac{\sin z}{(\cos z)'} \right|_{z = -\pi/2} = -1.\]
Поэтому
\[ \int\limits_{|z|=2} \tg z dz = -4\pi i.\]

\textbf{Пример 3.} Вычислить интеграл
\[ \int\limits_{x-i=3/2} \frac{e^{1/z^2}}{z^2+1} dz.\]

\underline{\textit{Решение}}. В области D: $|z-i| < \frac{3}{2}$ функция $f(z) = \frac{e^{1/z^2}}{z^2+1}$ имеет две особые точка: z = i -- после первого порядка и z = 0 --  существенно особая точка.

По формуле (5) из \S 9 имеем 
\[ \left.res f(i) = \frac{e^{1/z^2}}{2z}\right|_{z=i} = \frac{e^{-1}}{2i}\]
Для нахождения вычета точки $z = 0$ необходимо иметь лорановское разложение функции f(z) в окрестности точка $z = 0$. Однако в данном случае искать ряд Лорана нет необходимости: функция f(z) четная, и поэтому можно заранее сказать, что в ее лорановском разложении будут содержаться только четные степени z и $\frac{1}{z}$. Так что $c_{-1} = 0$ и, следовательно,
\[ resf(0) = 0\]
По теореме Коши о вычетах 
\[ \int\limits_{|x-i|=3/2} \frac{e^{1/z^2}}{z^2 + 1} dz = \frac{\pi}{e  }.\]

\textbf{Пример 4.} Вычислить интеграл
\[\int\limits_{|z| = 2} \frac{1}{z-1} \sin \frac{1}{z} dz.\]

\underline{\textit{Решение}}. В круге $|z| \leq 2$ подынтегральная функция имеет две особые точки $z = 1$ и $z = 0$. Легко установить, что $z = 1$ есть простой полюс, поэтому 
\[ res_{z=1} \left(\frac{1}{z-1} \sin \frac{1}{z} \right) = \left. \frac{\sin \frac{1}{z}}{(z-1)'}\right|_{z=1} = \sin 1.\]

Для установления характера особой точки $z = 0$ напишем ряд Лорана для функции $\frac{1}{z-1} \sin\frac{1}{z}$ в окрестности этой точки. Имеем
\[\frac{1}{z-1} \sin \frac{1}{z} = -\frac{1}{1-z} \sin \frac{1}{z} = (-1+z+z^2+...)\left(\frac{1}{z} - \frac{1}{3!z^3} + \frac{1}{5!z^5}-...\right)\]
\[ = -\left(1 - \frac{1}{3} + \frac{1}{5!} - ...\right) \frac{1}{z} + \frac{c_{-2}}{z^2} + \frac{c_{-3}}{z^3}+...+\]
\[c_{-k} \neq 0, k = 2,3,... .\]
Так как ряд Лорана содержит бесконечное множество членов с отрицательными степенями z, то точка $z = 0$ является существенно особой. Вычет подынтегральной функции в этой точке равен
\[ res_{z=0} \frac{\sin \frac{1}{z}}{z-1} = c_{-1} = -\left(1 - \frac{1}{3!} + \frac{1}{5!} - ...\right) = - \sin 1.\]
Следовательно,
\[ \int\limits_{|z|=2} \frac{1}{z-1} \sin \frac{1}{z} dz = 2\pi i(\sin 1 - \sin 1) = 0.\]
\subsection*{\textbf{\large Задачи для самостоятельного решения}}
\noindent\rule{\textwidth}{1pt}
Вычислить интегралы:\\[0.33cm]
\textbf{347.} $\int\limits_{|z|=1} z \tan\pi z dz.$
\hspace{0.7cm}
\textbf{348.} $\int\limits_{C} \frac{z dz}{(z-1)^2(z+2)}$, \hspace{1cm} где $C: x^{2/3} + y^{2/3}=3^{2/3}$.\\[0.7cm]
\textbf{349.} $\int\limits_{|z|=2} \frac{e^x dz}{z^3(z+1)}$.
\hspace{1.4cm}
\textbf{350.} $\int\limits_{|z-i|=3} \frac{e^{x^2}-1}{x^3-iz^2}dz$.
\hspace{0.6cm}
\textbf{351.} $\int\limits_{|z|=1/2}z^2\sin\frac{1}{z}dz$
\\[0.7cm]
\textbf{352.} $\int\limits_{|z|=\sqrt{3}}\frac{\sin\pi z}{z^2-z}dz$.
\hspace{1.07cm}
\textbf{353.} $\int\limits_{|z+1|=4}\frac{z dz}{e^x+3}$.
\hspace{1.2cm}
\textbf{354.} $\int\limits_{|z|=1}\frac{z^2dz}{\sin ^3z\cos z}$.\\[0.7cm]
\textbf{355.} $\int\limits_{|x-i|=1}\frac{e^zdz}{z^4+2z^2+1}$.
\hspace{0.79cm}
\textbf{356.} $\int\limits_{|z|=4}\frac{e^{iz}dz}{(z-\pi)^3}$.\\[0.7cm]
\textbf{357.} $\int\limits_{C}\frac{\cos\frac{z}{2}}{z^2-4}dz$,
\hspace{0.7cm}
$C: \frac{x^2}{4}+y^2=1$.
\hspace{0.35cm}
\textbf{358.} $\int\limits_{C}\frac{e^{2z}}{z^3-1}dz$,
\hspace{0.3cm}
$C: x^2+y^2-2x=0$.\\[0.7cm]
\textbf{359.} $\int\limits_{C}\frac{\sin\pi z}{(z^2-1)^2}dz$,
\hspace{0.3cm}
$C: \frac{x^2}{4}+y^2=1$.
\hspace{0.2cm}
\textbf{360.} $\int\limits_{C}\frac{z+1}{z^2+2z-3}dz$,
\hspace{0.3cm}
$C: x^2+y^2=16.$
\end{document}
